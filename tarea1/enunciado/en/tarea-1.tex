\documentclass[english, spanish, fleqn]{article}
\usepackage{fourier}
\usepackage{babel}
\usepackage{minted}
\usepackage[utf8]{inputenc}
\usepackage{csquotes}
\usepackage[colorlinks, urlcolor=blue]{hyperref}

\newcommand{\num}{1}

\title{Compiladores \\
       ``Le hacemos al léxico''}
\author{Dragon Slayers}
\date{8 de octubre de 2018}

\begin{document}
\maketitle
\thispagestyle{empty}

  El rendimiento del análisis léxico es vital.
  Compararemos varias opciones para efectuar una tarea típica:
  aislar \textquote{palabras} en archivos grandes
  (similar a lo que hace el programa \mintinline{shell}{wc(1)}).
  Para nuestros efectos,
  una palabra es una secuencia de letras, posiblemente unidas por guión
  (como son \texttt{letrado} o \texttt{no-terminal};
   note que por ejemplo \texttt{O'Higgins} cuenta como dos palabras,
   \texttt{Entel123} o \texttt{3Com} cuentan como una
   y \texttt{1234} o \texttt{3,141592} no cuentan).
  Para simplificar,
  considere texto puramente ASCII
  (texto internacional
   tiene sus propias enternecedoras complicaciones\ldots).
  Use las siguientes alternativas:
  \begin{enumerate}
  \item % 20182t1p1
    Escrito a mano,
    leyendo caracter a caracter vía \mintinline{C}{getc(3)}.
  \item % 20182t1p2
    Escrito a mano,
    leyendo el archivo completo a memoria con \mintinline{C}{read(2)}
    y procesando allí
  \item % 20182t1p3
    Escrito a mano,
    mapeando el archivo completo a memoria con \mintinline{C}{mmap(2)}
    y procesando allí
  \item % 2082t1p4
    Usando \mintinline{shell}{flex(1)},
    un \emph{\foreignlanguage{english}{scanner}} interactivo
  \item % 2082t1p5
    Usando \mintinline{shell}{flex(1)},
    un \emph{\foreignlanguage{english}{scanner}} no interactivo
  \end{enumerate}
  Resuma sus resultados y conclusiones.

\input{condiciones}
  \vfill\hfill DS/HvB/\LaTeXe
\end{document}

%%% Local Variables:
%%% mode: latex
%%% TeX-master: t
%%% End:
