\documentclass[english, spanish, fleqn]{article}
\usepackage{fourier}
\usepackage{babel}
\usepackage{minted}
\usepackage[utf8]{inputenc}
\usepackage{csquotes}
\usepackage[colorlinks, urlcolor=blue]{hyperref}

\newcommand{\num}{1}

\title{Compiladores \\
       ``Le hacemos al léxico''}
\author{Dragon Slayers}
\date{8 de octubre de 2018}

\begin{document}
\maketitle
\thispagestyle{empty}

  El rendimiento del análisis léxico es vital.
  Compararemos varias opciones para efectuar una tarea típica:
  aislar \textquote{palabras} en archivos grandes
  (similar a lo que hace el programa \mintinline{shell}{wc(1)}).
  Para nuestros efectos,
  una palabra es una secuencia de letras, posiblemente unidas por guión
  (como son \texttt{letrado} o \texttt{no-terminal};
   note que por ejemplo \texttt{O'Higgins} cuenta como dos palabras,
   \texttt{Entel123} o \texttt{3Com} cuentan como una
   y \texttt{1234} o \texttt{3,141592} no cuentan).
  Para simplificar,
  considere texto puramente ASCII
  (texto internacional
   tiene sus propias enternecedoras complicaciones\ldots).
  Use las siguientes alternativas:
  \begin{enumerate}
  \item % 20182t1p1
    Escrito a mano,
    leyendo caracter a caracter vía \mintinline{C}{getc(3)}.
  \item % 20182t1p2
    Escrito a mano,
    leyendo el archivo completo a memoria con \mintinline{C}{read(2)}
    y procesando allí
  \item % 20182t1p3
    Escrito a mano,
    mapeando el archivo completo a memoria con \mintinline{C}{mmap(2)}
    y procesando allí
  \item % 2082t1p4
    Usando \mintinline{shell}{flex(1)},
    un \emph{\foreignlanguage{english}{scanner}} interactivo
  \item % 2082t1p5
    Usando \mintinline{shell}{flex(1)},
    un \emph{\foreignlanguage{english}{scanner}} no interactivo
  \end{enumerate}
  Resuma sus resultados y conclusiones.

% Condiciones generales de tareas de Estructuras Discretas, 2015
\section{Condiciones de entrega}

  \begin{itemize}
  \item
    La tarea se realizará \emph{individualmente}
    (esto es grupos de una persona),
    sin excepciones.
  \item
    La entrega debe realizarse vía \href{http://moodle.inf.utfsm.cl}{Moodle}
    en un \emph{tarball} en el área designada al efecto, bajo el formato
    \texttt{tarea-\num-\emph{rol}.tar.gz}
    (\texttt{rol} con dígito verificador y sin guión).

    Dicho \emph{tarball} debe contener las fuentes en LaTeX
    (al menos \texttt{tarea.tex})
    de la parte escrita de su entrega,
    además de un archivo \texttt{tarea-\num.pdf},
    correspondiente a la compilación de esas fuentes.
  \item
    En  caso de haber programas,
    su ejecutable \emph{debe} llamarse \texttt{tarea-\num},
    de haber varias preguntas solicitando programas,
    estos deben llamarse \texttt{tarea-\num-1},
    \texttt{tarea-\num-2},
    etc.
    Si hay programas compilados,
    incluya una \texttt{Makefile}
    que efectúe las compilaciones correspondientes.

    Los programas se evalúan según que tan claros
    (bien escritos)
    son,
    si se compilan y ejecutan sin errores o advertencias según corresponda.
    Parte del puntaje es por ejecución correcta con casos de prueba.
    Si el programa no se ciñe a los requerimientos de entrada y salida,
    la nota respectiva es cero.
  \item
    Además de esto,
    la parte escrita de la tarea debe en hojas de tamaño carta
    en Secretaría Docente de Informática (Piso 1, edificio F3).
  \item
    Tanto el \emph{tarball} como la entrega física
    deben realizarse el día indicado
    en \href{http://moodle.inf.utfsm.cl}{Moodle}.
    No entregar la parte escrita en papel
    o no entregar en formato electrónico tiene un descuento de 50 puntos.

    Por cada día de atraso se descontarán 20 puntos.
    A partir del tercer día de atraso
    no se reciben más tareas,
    y la nota de la tarea es cero.
  \item
    Nos reservamos el derecho de llamar a interrogación
    sobre algunas de las tareas entregadas.
    En tal caso,
    la nota base
    (antes de descuentos por atraso y otros)
    es la de la interrogación.
    No presentarse a la interrogación sin justificación previa
    significa automáticamente nota cero.
  \end{itemize}

  \vfill\hfill DS/HvB/\LaTeXe
\end{document}

%%% Local Variables:
%%% mode: latex
%%% TeX-master: t
%%% End:
