\documentclass[spanish, fleqn]{article}
\usepackage[spanish]{babel}
\usepackage[utf8]{inputenc}
\usepackage{amsmath, amssymb}
%\usepackage{fancyhdr}
%\usepackage{lipsum}
%\usepackage{enumerate}
%\usepackage{multicol}
\usepackage[colorlinks, urlcolor=blue]{hyperref}
\usepackage{graphicx}
\graphicspath{ {images/} }
\usepackage{xcolor}
\usepackage{verbatim}
\usepackage{listings}
\usepackage{ mathrsfs }
\usepackage{wrapfig}
\usepackage{enumitem}
\usepackage{ dsfont }
\usepackage{afterpage}

\usepackage{tikz} %% Grafos
\usepackage{fancyhdr} %% Para las cosas en las esquinas de cada pagina

\usepackage[a4paper,bindingoffset=0.0in,left=0.60in,right=0.70in,top=0.7in,bottom=0.7in,footskip=.25in]{geometry}

\newcommand\tab[1][0.5cm]{\hspace*{#1}}
\newcommand{\overbar}[1]{\mkern 1.5mu\overline{\mkern-1.5mu#1\mkern-1.5mu}\mkern 1.5mu}

\newcommand{\cero}{0}
\newcommand{\sumMayorIgual}[2][\cero]{\sum_{n\geq#1}{#2}}

\newcommand{\eclosure}[1]{\epsilon\text{-closure}(\{#1\})}

\newcommand\blankpage{%
    \null
    \thispagestyle{empty}%
    \addtocounter{page}{-1}%
    \newpage}
    
\definecolor{bggray}{rgb}{0.95,0.95,0.95}
\lstdefinestyle{mypy}{
  language=python,
  backgroundcolor=\color{bggray},
  basicstyle=\ttfamily\small\color{orange!70!black},
  frame=L,
  keywordstyle=\bfseries\color{green!40!black},
  commentstyle=\itshape\color{purple!40!black},
  identifierstyle=\color{blue},
  stringstyle=\color{red},
  numbers=left,
  showstringspaces=false,
  xrightmargin=5pt,
  xleftmargin=10pt
}
\lstdefinestyle{myC}{
  language=C,
  backgroundcolor=\color{bggray},
  basicstyle=\ttfamily\small\color{orange!70!black},
  frame=L,
  keywordstyle=\bfseries\color{green!40!black},
  commentstyle=\itshape\color{purple!40!black},
  identifierstyle=\color{blue},
  stringstyle=\color{red},
  numbers=left,
  showstringspaces=false,
  xrightmargin=5pt,
  xleftmargin=10pt
}
\lstdefinestyle{mygeneric}{
  backgroundcolor=\color{bggray},
  basicstyle=\ttfamily\small\color{orange!70!black},
  frame=L,
  keywordstyle=\bfseries\color{green!40!black},
  commentstyle=\itshape\color{purple!40!black},
  identifierstyle=\color{blue},
  stringstyle=\color{red},
  numbers=left,
  showstringspaces=false,
  xrightmargin=5pt,
  xleftmargin=10pt
}
\renewcommand{\lstlistingname}{Código}

\newcommand{\ramo}{INF-341: Compiladores}
\newcommand{\numeroTarea}{1}
\newcommand{\nombreTarea}{``Le hacemos al léxico''}

\newcommand{\xstar}{x^\star}

\begin{document}

\lstset
{
    language=[LaTeX]TeX,
    breaklines=false,
    basicstyle=\tt\scriptsize,
    keywordstyle=\color{blue},
    identifierstyle=\color{magenta},
}

\title{\ramo \\
Tarea \#\numeroTarea\\
\nombreTarea
}
\author{\href{mailto:anghelo.carvajal.14@sansano.usm.cl}{Anghelo Carvajal} \\ 201473062-4
}

\maketitle

\pagenumbering{gobble} %% Desactiva la numeracion de paginas


\pagestyle{fancy}
\fancyhf{}
\lhead{Tarea \numeroTarea: \nombreTarea}

%% \chead{\thesection} % 
\chead{\ramo}

%% \rhead{\rightmark} % 
\rhead{\rightmark}

\lfoot{\LaTeXe}

%% \cfoot{} %

\rfoot{Página \thepage}

%  \thepage
% adds number of the current page.
%  \thechapter
% adds number of the current chapter.
%  \thesection
% adds number of the current section.
%  \chaptername
% adds the word "Chapter" in English or its equivalent in the current language.
%  \leftmark
% adds name and number of the current top-level structure (for example, Chapter for reports and books classes; Section for articles ) in uppercase letters.
%  \rightmark
% adds name and number of the current next to top-level structure (Section for reports and books; Subsection for articles) in uppercase letters.

\newpage

%% \section{Section}

\pagenumbering{arabic} %% Activa numeracion de paginas

Para poder crear el analizador léxico, crearemos una gramática que acepte palabras. Para que una palabra sea aceptada como tal, esta debe contener al menos una letra (minúscula o mayúscula). Una palabra puede contener caracteres numéricos o guiones. Esta gramática quedaría formada de la siguiente forma:
\begin{align*}
    LETRA &= [a-zA-Z] \\
    DEMAS &= [0-9]|"-" \\
    Gramatica &= (\{DEMAS\}|\{LETRA\})^*\{LETRA\}^+(\{DEMAS\}|\{LETRA\})^*
\end{align*}

Tanto como para la parte 1 (leer el archivo con \texttt{getc(3)}), la parte 2 (usar \texttt{read(2)}) y la parte 3 (\texttt{mmap(2)}), el analizador léxico es el mismo. Se abre el archivo, saltan los caracteres no validos, hasta encontrar algún carácter valido. Se lee y analiza para saber si es una palabra valida y se contabiliza si es el caso. Luego se sigue leyendo y saltando caracteres inválidos hasta llegar al fin del archivo y finalmente imprimir por pantalla el conteo de palabras.

La implementación del código descrito seria la siguiente:

\lstinputlisting[
    style=myC,
    caption=\texttt{main.c}
    ] {src/common/main.c}
    
La función \texttt{getFile()} se encarga de abrir el archivo de la forma correspondiente. \texttt{closeFile()} se encarga de cerrar el archivo (o vaciar la memoria según corresponda). \texttt{getCharacter()} entrega el siguiente carácter a leer (o \texttt{EOF} según corresponda).

La implementación con flex es bastante sencilla. Tan solo implementar la gramática explicada anteriormente. La implementación seria la siguiente:

\lstinputlisting[
    style=mygeneric,
    caption=\texttt{4/tarea-1-4.l}
    ] {src/4/tarea-1-4.l}
    
Finalmente, analizamos los tiempos de ejecucion de cada uno de los programas y los comparamos. Para esta comparación, usaremos el programa \texttt{time(1)}.

Los resultados son los siguientes:
\begin{enumerate}
    \item \texttt{time ./tarea-1-1 MobyDick.txt} \\ \tab Tiempo de ejecucion: 29ms.
    
    \item \texttt{time ./tarea-1-2 MobyDick.txt} \\ \tab Tiempo de ejecucion: 24ms.
    
    \item \texttt{time ./tarea-1-3 MobyDick.txt} \\ \tab Tiempo de ejecucion: 19ms.
    
    \item \texttt{time ./tarea-1-4 < MobyDick.txt} \\ \tab Tiempo de ejecucion: 36ms.
    
    \item \texttt{time ./tarea-1-5 < MobyDick.txt} \\ \tab Tiempo de ejecucion: 32ms.
\end{enumerate}

Lógicamente, la solución mas rápida fue con \texttt{mmap(2)} debido al mapeo directo a la memoria virtual.

Las implementaciones usando flex, aunque mas rápidas de programar, son las mas lentas en \textit{runtime}, probablemente por el \textit{overhead} que produce todo el código auto-generado por flex.

La opción que utiliza \texttt{getc(3)} es la mas lenta (de las programadas a manos), debido a que lee carácter a carácter desde el disco. Leer desde la ram (parte 2, pasar el archivo a memoria con \texttt{read(2)}) es mas rápido, debido a que hay una única lectura del archivo.

\end{document}
