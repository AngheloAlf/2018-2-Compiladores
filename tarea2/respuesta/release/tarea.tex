\documentclass[spanish, fleqn]{article}
\usepackage[spanish]{babel}
\usepackage[utf8]{inputenc}
\usepackage{amsmath, amssymb}
%\usepackage{fancyhdr}
%\usepackage{lipsum}
%\usepackage{enumerate}
%\usepackage{multicol}
\usepackage[colorlinks, urlcolor=blue]{hyperref}
\usepackage{graphicx}
\graphicspath{ {images/} }
\usepackage{xcolor}
\usepackage{verbatim}
\usepackage{listings}
\usepackage{ mathrsfs }
\usepackage{wrapfig}
\usepackage{enumitem}
\usepackage{ dsfont }
\usepackage{afterpage}

\usepackage{tikz} %% Grafos
\usepackage{fancyhdr} %% Para las cosas en las esquinas de cada pagina

\usepackage[a4paper,bindingoffset=0.0in,left=0.60in,right=0.70in,top=0.7in,bottom=0.7in,footskip=.25in]{geometry}

\newcommand\tab[1][0.5cm]{\hspace*{#1}}
\newcommand{\overbar}[1]{\mkern 1.5mu\overline{\mkern-1.5mu#1\mkern-1.5mu}\mkern 1.5mu}

\newcommand{\cero}{0}
\newcommand{\sumMayorIgual}[2][\cero]{\sum_{n\geq#1}{#2}}

\newcommand{\eclosure}[1]{\epsilon\text{-closure}(\{#1\})}

\newcommand\blankpage{%
    \null
    \thispagestyle{empty}%
    \addtocounter{page}{-1}%
    \newpage}
    
\definecolor{bggray}{rgb}{0.95,0.95,0.95}
\lstdefinestyle{mypy}{
  language=python,
  backgroundcolor=\color{bggray},
  basicstyle=\ttfamily\small\color{orange!70!black},
  frame=L,
  keywordstyle=\bfseries\color{green!40!black},
  commentstyle=\itshape\color{purple!40!black},
  identifierstyle=\color{blue},
  stringstyle=\color{red},
  numbers=left,
  showstringspaces=false,
  xrightmargin=5pt,
  xleftmargin=10pt
}
\renewcommand{\lstlistingname}{Código}   
    

\newcommand{\numeroTarea}{2}
\newcommand{\nombreTarea}{``Le seguimos haciendo al léxico"}

\newcommand{\xstar}{x^\star}

\begin{document}

\lstset
{
    language=[LaTeX]TeX,
    breaklines=false,
    basicstyle=\tt\scriptsize,
    keywordstyle=\color{blue},
    identifierstyle=\color{magenta},
}

\title{INF-341: Compiladores \\
Tarea \#\numeroTarea\\
\nombreTarea
}
\author{\href{mailto:anghelo.carvajal.14@gmail.com}{Anghelo Carvajal} \\ 201473062-4
}

\maketitle

\pagenumbering{gobble} %% Desactiva la numeracion de paginas


\pagestyle{fancy}
\fancyhf{}
\lhead{Tarea \numeroTarea: \nombreTarea}

%% \chead{\thesection} % 

%% \rhead{\rightmark} % 
\rhead{\rightmark}

\lfoot{\LaTeXe}

%% \cfoot{} %

\rfoot{Página \thepage}

%  \thepage
% adds number of the current page.
%  \thechapter
% adds number of the current chapter.
%  \thesection
% adds number of the current section.
%  \chaptername
% adds the word "Chapter" in English or its equivalent in the current language.
%  \leftmark
% adds name and number of the current top-level structure (for example, Chapter for reports and books classes; Section for articles ) in uppercase letters.
%  \rightmark
% adds name and number of the current next to top-level structure (Section for reports and books; Subsection for articles) in uppercase letters.

\newpage

%% \section{Section}

\pagenumbering{arabic} %% Activa numeracion de paginas

\section{Introducción}

En nuestro analizador léxico para archivos \texttt{.ini}, definimos 8 tipos de tokens y 4 estados distintos de modo de poder identificar correctamente este archivo, para luego almacenar cada token en una estructura tipo diccionario.

El código mostrado son fragmentos del código completo final.

\section{Analizador léxico}

\subsection{Tokens}
Los tokens están definidos de la siguiente manera:
\begin{itemize}
    \item \texttt{NEWLINE}. \texttt{r"(\textbackslash n|\textbackslash r|(\textbackslash r\textbackslash n))"} \\ Calza los saltos de linea e indica que debemos entrar al estado inicial (\texttt{INITIAL}).
    
    Indicamos que retorna \texttt{None} para especificar que no nos interesa esto.
    
    \item \texttt{COMMENT}. \texttt{r"[;|\#].*"} \\ Calza los comentarios que empiecen con \texttt{;} o con \texttt{\#}. 
    
    Retorna nada, de modo que ignore los comentarios.
    
    \item \texttt{KEY}. \texttt{r"[\textasciicircum=;\textbackslash \#\textbackslash[\textbackslash]\textbackslash]+"} \\ Calza una llave o identificador. Este identificador esta definido como cualquier carácter que no sea un `\texttt{=}' (separador), `\texttt{;}', `\texttt{\#}' (comentarios) `\texttt{[}' o `\texttt{]}' (reservado para las secciones). 
    
    Además indicamos que debemos entrar al estado \texttt{KEY}. 
    
    Finalmente, tal como especifica la \href{http://portablecontacts.net/wiki/computing/ini-file}{documentación} proveída, tanto los identificadores como las secciones son \textit{case insensitivity}, de modo que retornamos el valor transformado a minúsculas.
    
    \item \texttt{KEY\_COMMENT}. \texttt{r"[ \textbackslash t]*[;|\textbackslash\#].*"} \\ Si estando en el estado \texttt{KEY} nos encontramos con un comentario, nos devolvemos al estado inicial.
    
    Si llegamos a este token, significa que el identificador fue \textit{seteado} sin un valor correspondiente, y tan solo se escribió un comentario al costado.
    
    \item \texttt{KEY\_NEWLINE}. \texttt{r"[ \textbackslash t]*(\textbackslash n|\textbackslash r|(\textbackslash r\textbackslash n))"}. \\ Similar al anterior, con la diferencia de que no hay comentarios al costado.
    
    \item \texttt{KEY\_SEPARATOR}. \texttt{r"[ \textbackslash t]*=[ \textbackslash t]*"} \\ Estando en el estado \texttt{KEY}, calza el símbolo `\texttt{=}' pudiendo estar rodeado de espacios o tabulaciones. 
    
    Después de esto entramos al estado \texttt{SEPARATOR}.
    
    \item \texttt{SEPARATOR\_VALUE}. \texttt{r"[\textasciicircum;\textbackslash \#\textbackslash n\textbackslash r]+"} \\ El valor asociado a un identificador. 
    
    Solo entramos aquí si es que antes hemos calzado un identificador y un separador.
    
    Puede ser cualquier carácter que no sean los reservados para comentarios, ni el salto de linea.
    
    Además, se quitan todos los espacios sobrantes al inicio o al final del valor. 
    
    También, si el valor es un numero entero positivo, este es transformado a una variable del tipo \texttt{int}, y si es un valor booleano (\texttt{true} o \texttt{false}, \texttt{case insensitive}) lo convierte a una variable del tipo \texttt{bool}. 
    
    Finalmente indica de que debemos devolvernos al estado \texttt{INITIAL}.
    
    \item \texttt{SECTION}. \texttt{r"\textbackslash["} \\ Calza el caracter [. Luego indicamos que entramos al estado \texttt{SECTION}
    
    \item \texttt{SECTION\_DATA}. \texttt{r"[a-zA-Z\textbackslash d\textbackslash-\textbackslash.\_]+"} \\ El nombre de la sección. Solo entramos aquí si ya calzamos un `\texttt{[}' con anterioridad.
    
    Esto calza caracteres alfanuméricos, guiones (-), puntos (.) y guiones bajos (\_).
    
    Luego entramos al estado \texttt{DATA}.
    
    \item \texttt{DATA\_END}. \texttt{r"\textbackslash ]"} \\ El final de una sección. 
    
    Solo entramos aquí si es que ya encontramos un nombre para la sección. Esto es útil para evitar secciones con nombres vacíos.
    
    Luego nos vamos al estado \texttt{INITIAL}.
\end{itemize}

\subsection{Estados}
Se definieron 4 estados, complementando a \texttt{INITIAL}. Se detallan a continuación:

\begin{itemize}
    \item \texttt{KEY}. Indica que hemos calzado un identificador.
    
    Estando en este estado, solo podemos consumir los símbolos de comentarios o nueva linea, lo cual nos llevaría al estado \texttt{INITIAL}; o consumir el símbolo separador, lo cual nos lleva al estado \texttt{SEPARATOR}.
    
    \item \texttt{SEPARATOR}. Indica que hemos calzado el separador.
    
    Solo podemos entrar a este estado si es que ya nos encontramos en el estado \texttt{KEY}.
    
    En este estado, solo podemos consumir lo correspondiente al valor de un identificador, lo cual nos llevara al estado inicial nuevamente. Consumir cualquier otra cosa es un error.
    
    \item \texttt{SECTION} Indica que hemos calzado el símbolo de inicio de sección.
    
    En este estado solo podemos calzar el nombre de la sección, lo cual debería llevarnos al estado \texttt{DATA}
    
    \item \texttt{DATA}. Indica que hemos calzado el nombre de la sección.
    
    Estando en este estado, solo podemos calzar el final de la sección, llevándonos al estado \texttt{INITIAL}.
\end{itemize}

\subsection{La implementación}
A continuación, se puede ver el código del analizador léxico. Se han omitido las funciones que manejan errores.

\lstinputlisting[
    style  = mypy,
    caption= \texttt{analizador\_lexico.py}]{lexico.py}
    

\section{Hash}

La función que se encarga de transformar los datos de entrada a una estructura \texttt{hash}, en este caso un diccionario, es \texttt{parseIniFile(fullFile)}, donde \texttt{fullFile} es toda la entrada de texto.

El diccionario retornado tiene la siguiente estructura:
\begin{itemize}
    \item Las llaves son los nombres de las secciones en tipo \texttt{string}. 
    
    Si un identificador esta definido al comienzo del documento sin una sección, este sera almacenado dentro de la sección \texttt{""} (string vació). 
    
    Si todos los identificadores están dentro de alguna sección, la sección \texttt{""} no existirá dentro del diccionario.
    
    \item Los valores son diccionarios que contienen los identificadores de la sección correspondiente.
    
    Esta estructurado de la siguiente forma:
    \begin{itemize}
        \item La llave es el nombre del identificador. Tipo \texttt{string}.
        
        \item El valor es el valor del identificador correspondiente.
        
        Puede ser de distintos tipos, siendo \texttt{bool} si se encontro un \texttt{true} o un \texttt{false}, \texttt{int} si se encontro unicamente un numero positivo, \texttt{None} si el identificador fue \textit{seteado} sin un valor acompañante, o tipo \texttt{string} en caso contrario a los otros.
    \end{itemize}
\end{itemize}

\lstinputlisting[
    style  = mypy,
    caption= \texttt{parser\_ini.py}]{parse_ini.py}




\end{document}
